\documentclass[../main.tex]{subfiles}

\begin{document}
\section{Experiments - Task2}

% Perhaps this should be moved if the same technique is used in other places
It is essential to note here that the two cost type criteria: \emph{buying} and \emph{maint}
have been transformed in the following way:
\verb|new_value = 1 - old_value|.
Thanks to this procedure more preferred values are represented by larger numbers,
just as in the case of gain type criteria.
This is done in order to facilitate using non-decreasing marginal value functions for all criteria.

% Based on the parameters obtained, can we say something about the user’s
% preferences? Are there any criteria that have no effect, or have a decisive
% influence? Whether there are any preference thresholds? Are there any
% evaluations on criteria that are indifference in terms of preferences?
No criterion seems to have a decisive influence or be irrelevant.

Their maximum values of their marginal value functions ("weights")
are all concentrated roughly between 0.14 and 0.2.

All but one function are more or less flat in the larger values of $g_i$.
The one exception - \emph{buying}. Its marginal value function is particularly interesting.

It is seemingly composed out of two linear pieces, with a single breakpoint
somewhere between 0.2 and 0.4, towards the beginning of this segment.
This likely reflects the threshold between \emph{vhigh} and \emph{high}.

Having 3 doors seems to be a very big advantage over having only 2,
but adding more doesn't give much benefit

\end{document}
