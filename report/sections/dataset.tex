\documentclass[../main.tex]{subfiles}

\begin{document}

\section{Dataset}
We chose the \textbf{car evaluation} dataset from the \emph{UCI Machine
Learning
Repository}\footnote{\url{https://archive.ics.uci.edu/ml/datasets/Car+Evaluation}},
due to its simplicity and alignment with the project specification.

\subsection{Overview}
The dataset comprises of \textbf{1728 instances}, each having \textbf{6
features} and belonging to one of \textbf{4 classes}. Each feature is
monotonic, with some being categorical, and some numerical.

\subsection{Transformation}
In order to make the data suitable for the project, it was necessary to
transform the features and classes to monotonic numerical values and normalize
them. This was done in the following way:
\begin{longtable}[c]{|c|c|c|c|}
	\hline
	\multirow{2}*{\textbf{feature}} & \multirow{2}*{\textbf{type}} & \multicolumn{2}{c|}{\textbf{values}} &
	\cline{3-4}
	& & \textbf{before} & \textbf{after} \\
	\hline
	\endfirsthead
	\hline
	\multirow{2}*{\textbf{feature}} & \multirow{2}*{\textbf{type}} & \multicolumn{2}{c|}{\textbf{values}} &
	\cline{3-4}
	& & \textbf{before} & \textbf{after} \\
	\hline
	\endhead
	\multirow{4}{*}{\emph{buying}, \emph{maint}} & \multirow{4}*{cost}
		  & \verb`low` & \verb`0.000000` \\
		& & \verb`med` & \verb`0.333333` \\
		& & \verb`high` & \verb`0.666667` \\
		& & \verb`vhigh` & \verb`1.000000` \\
	\hline
	\multirow{4}{*}{\emph{doors}} & \multirow{4}*{gain}
		  & \verb`2` & \verb`0.000000` \\
		& & \verb`3` & \verb`0.333333` \\
		& & \verb`4` & \verb`0.666667` \\
		& & \verb`5more` & \verb`1.000000` \\
	\hline
	\multirow{3}{*}{\emph{persons}} & \multirow{3}*{gain}
		& \verb`2` & \verb`0.000000` \\
		& & \verb`4` & \verb`0.500000` \\
		& & \verb`more` & \verb`1.000000` \\
	\hline
	\multirow{3}{*}{\emph{lug\_boot}} & \multirow{3}*{gain}
		& \verb`small` & \verb`0.000000` \\
		& & \verb`med` & \verb`0.500000` \\
		& & \verb`big` & \verb`1.000000` \\
	\hline
	\multirow{3}{*}{\emph{safety}} & \multirow{3}*{gain}
		& \verb`low` & \verb`0.000000` \\
		& & \verb`med` & \verb`0.500000` \\
		& & \verb`high` & \verb`1.000000` \\
	\hline
	\hline
	\multirow{4}{*}{\textbf{class}} & \multirow{4}*{--}
		& \verb`unacc` & \verb`1.000000` \\
		& & \verb`acc` & \verb`2.000000` \\
		& & \verb`good` & \verb`3.000000` \\
		& & \verb`vgood` & \verb`4.000000` \\
	\hline
\end{longtable}

\noindent
Note that for some tasks, the decision class will be binarized to represent
whether a car is \textbf{at least acceptable}
(\verb`unacc → 0.000000`, everything else \verb`→ 1.000000`).

\subsection{Sample}

(the first columns enumerate rows and are not part of the data)

\begin{longtable}[c]{|r|n|n|n|n|n|n|l|}
	\endhead
	\hline
	1 & vhigh & vhigh & 2 & 2 & med & med & unacc \\
	2 & high & low & 5more & 2 & small & low & unacc \\
	3 & med & low & 4 & 4 & med & high & vgood \\
	4 & low & low & 4 & more & med & med & good \\
	5 & vhigh & vhigh & 5more & more & med & med & unacc \\
	\hline
	\caption{Raw sample, as present in the original CSV file}
\end{longtable}

\begin{longtable}[c]{|r|n|n|n|n|n|n|l|}
	\hline
	\emph{\#} & \emph{buying} & \emph{maint} & \emph{doors} & \emph{persons} & \emph{lug\_boot} & \emph{safety} & \emph{class} \\
	\hline
	\endfirsthead
	\hline
	\emph{\#} & \emph{buying} & \emph{maint} & \emph{doors} & \emph{persons} & \emph{lug\_boot} & \emph{safety} & \emph{class} \\
	\hline
	\endhead
	1 & 1.000000 & 1.000000 & 0.000000 & 0.000000 & 0.500000 & 0.500000 & 0.000000 \\
	2 & 0.666667 & 0.000000 & 1.000000 & 0.000000 & 0.000000 & 0.000000 & 0.000000 \\
	3 & 0.333333 & 0.000000 & 0.500000 & 0.500000 & 0.500000 & 1.000000 & 4.000000 \\
	4 & 0.000000 & 0.000000 & 0.500000 & 1.000000 & 0.500000 & 0.500000 & 3.000000 \\
	5 & 1.000000 & 1.000000 & 1.000000 & 1.000000 & 0.500000 & 0.500000 & 0.000000 \\
	\hline
	\caption{Processed sample}
\end{longtable}

The processed dataset is available at \verb`data/car-evaluation.csv` (from the
project root directory).

\subsection{Splitting into train and test sets}
In order to be able to compare all models fairly, it is essential to guarantee
that their respective training and testing sets are identical. This is achieved
with SciKit-Learn's \verb`train_test_split()` function, by fixing its random
seed. \verb`train_test_split()` also takes care of the stratification of the
resulting sets.

\subsection{The "3-alternative analyses"}
For all models' "3-alternative analyses" we used the same 3 alternatives
(again, to retain fairness of comparison):

\begin{longtable}[c]{|r|n|n|n|n|n|n|l|}
	\hline
	\emph{\#} & \emph{buying} & \emph{maint} & \emph{doors} & \emph{persons} & \emph{lug\_boot} & \emph{safety} & \emph{class} \\
	\hline
	\endfirsthead
	\hline
	\emph{\#} & \emph{buying} & \emph{maint} & \emph{doors} & \emph{persons} & \emph{lug\_boot} & \emph{safety} & \emph{class} \\
	\hline
	\endhead
	1    & 1.000000 & 1.000000 & 0.000000 & 0.000000 & 0.000000 & 0.000000 & 1.000000 \\
	1636 & 0.000000 & 0.000000 & 0.000000 & 0.500000 & 1.000000 & 0.000000 & 1.000000 \\
	1728 & 0.000000 & 0.000000 & 1.000000 & 1.000000 & 1.000000 & 1.000000 & 4.000000 \\
	\hline
	\caption{Alternatives chosen for all 3-alternative analyses}
\end{longtable}
As you can see, the first alternative represents the worst case, the third the
best case, and the second one lies somewhere in the middle.

\end{document}
