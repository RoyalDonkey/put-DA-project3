\documentclass[../main.tex]{subfiles}

\begin{document}

\section{Dataset}
We chose the \textbf{breast cancer} dataset from \emph{Padeborn University
Monotone Learning
Datasets}\footnote{\url{https://en.cs.uni-paderborn.de/is/research/research-projects/software/monotone-learning-datasets}},
due to its simplicity and alignment with the project specification.

\subsection{Dataset exploration}
\paragraph{Overview} The dataset comprises of \textbf{278 instances}, each having \textbf{7 features}
and belonging to one of \textbf{2 classes}. Each feature is monotonic and
normalized to the range $\left[ 0; 1 \right]$ (both inclusive). The decision
classes are represented by the values $1.0$ and $2.0$.

\paragraph{Feature \& class names} The data lacks column names, which makes it
impossible to know what each feature represents and what the classes mean. This
does not hinder experiment results, because in our case the meaning behind the
numbers don't play an important role in running and testing different
classifiers. However, for the sake of analysis, we will refer to features as
\emph{F1} -- \emph{F7} and treat classes $1.0$ and $2.0$ as \emph{malignant} and
\emph{benign}, respectively.

\subsection{Dataset sample}

(the first columns enumerate rows and are not part of the data)

\begin{longtable}[c]{|r|n|n|n|n|n|n|n|l|}
	\endhead
	\hline
	1 & 1.000000 & 0.500000 & 0.000000 & 1.000000 & 1.000000 & 1.000000 & 0.000000 & 2.000000 \\
	2 & 0.500000 & 0.500000 & 0.000000 & 0.000000 & 0.000000 & 1.000000 & 0.000000 & 1.000000 \\
	3 & 0.500000 & 0.833333 & 0.000000 & 0.000000 & 0.500000 & 0.000000 & 0.000000 & 2.000000 \\
	4 & 1.000000 & 0.833333 & 0.000000 & 1.000000 & 1.000000 & 1.000000 & 1.000000 & 1.000000 \\
	5 & 1.000000 & 1.000000 & 0.125000 & 1.000000 & 0.500000 & 0.000000 & 0.000000 & 2.000000 \\
	\hline
	\caption{Raw sample, as present in the CSV file}
\end{longtable}

\begin{longtable}[c]{|r|n|n|n|n|n|n|n|l|}
	\hline
	\emph{\#} & \emph{F1} & \emph{F2} & \emph{F3} & \emph{F4} & \emph{F5} & \emph{F6} & \emph{F7} & \emph{class} \\
	\hline
	\endfirsthead
	\hline
	\emph{\#} & \emph{F1} & \emph{F2} & \emph{F3} & \emph{F4} & \emph{F5} & \emph{F6} & \emph{F7} & \emph{class} \\
	\hline
	\endhead
	1 & 1.000000 & 0.500000 & 0.000000 & 1.000000 & 1.000000 & 1.000000 & 0.000000 & benign\\
	2 & 0.500000 & 0.500000 & 0.000000 & 0.000000 & 0.000000 & 1.000000 & 0.000000 & malignant\\
	3 & 0.500000 & 0.833333 & 0.000000 & 0.000000 & 0.500000 & 0.000000 & 0.000000 & benign \\
	4 & 1.000000 & 0.833333 & 0.000000 & 1.000000 & 1.000000 & 1.000000 & 1.000000 & malignant\\
	5 & 1.000000 & 1.000000 & 0.125000 & 1.000000 & 0.500000 & 0.000000 & 0.000000 & benign \\
	\hline
	\caption{Processed sample}
\end{longtable}

The full dataset is available at \verb`data/breast-cancer.csv` (from the project root directory).

\end{document}
